\documentclass[14pt,a4paper,oneside]{report}

\usepackage{pdfpages}
\usepackage{epstopdf}
\usepackage{url}
\usepackage{listings}
\usepackage{subfig}


\renewcommand{\thesection}{\arabic{section}}
\setcounter{secnumdepth}{3}

\title{\textbf{Optimal decision making for complex problems}
\\ Lunar Lander}

\author{Francois Delarbre \and Simon Lorent}


\begin{document}

\vfill
\date{Academic year 2017 - 2018}

\maketitle

\section{Introduction}
\paragraph{} In this project, we have chose to try to solve the lunar lander\footnote{\url{https://gym.openai.com/envs/LunarLander-v2/}} probleme from openAi gym. To do so, we have tryied two methods, the first one, by adapting code found\footnote{\url{https://www.superdatascience.com/artificial-intelligence/}} for an other problem, which use deep convolutional Q-learning. For the second one we tryied to implement A3C algorithm by ourself. 

\section{Deep Convolutional Q-learning}
\paragraph{} Deep convolutional Q-learning, as the name says, make use of a convolutional neural network, which take as an input images of the problem, and output the action to take. 
\end{document}
